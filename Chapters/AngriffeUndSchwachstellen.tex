\chapter{Angriffe und Schwachstellen}
Fokus in diesem Kapitel liegt auf der Obersten Schicht von Webdiensten, der Applikation selbst.
\textbf{Unterliegende Komponenten dürfen trotzdem nie vernachlässigt werden!}

\paragraph{OWASP}
Open Web Application Security Project: Gemeinnützige Organisation mit dem Ziel sicherer Webanwendungen.

\section{Injection-Schwachstellen}
\begin{itemize}
	\item entstehen, wenn Eingaben eines Clients \textbf{ungeprüft an einen Interpreter} weitergeleitet werden
	\item einfach auszunutzen, weit verbreitet, verursachen hohen Schaden
	\item \textbf{Vermeidung:} 
	\begin{itemize}
	\item Sichere APIs (z.B. Prepared Statement)
	\item Escaping, um Metazeichen der jeweiligen Syntax zu \enquote{entschärfen}
	\item White-Lists bei Eingaben
	\end{itemize}
\end{itemize}

\paragraph{Beispiel ShellShock}
\begin{itemize}
	\item Linux-Shell bash erlaubt das Definieren von Funktionen, die an Sub-Shells weitergegeben werden können. 
	\item \textbf{Schwachstelle:} Stehen hinter der Funktionsdefinition Befehle, dann werden die beim Start der Shell ausgeführt!
\end{itemize}